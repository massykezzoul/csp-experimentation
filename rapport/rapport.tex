\documentclass[oneside,13pt,a4paper]{report}

% Chargement d'extensions
\usepackage[utf8]{inputenc}
\usepackage[french]{babel}
\usepackage{graphicx}
\usepackage[top=3cm, bottom=3cm, left=3cm, right=3cm]{geometry}
\usepackage{amsmath}
\usepackage{amssymb}

% Liens et autres
\usepackage{hyperref}
\hypersetup{
    colorlinks=true,
    linkcolor=black,
	urlcolor=blue,
	pdftitle={Rapport de projet Claims checking},
	bookmarks=true,
}

% Bout de code
\usepackage{listings}
\usepackage{color}
%\lst{

%}

\definecolor{mygreen}{rgb}{0,0.6,0}
\definecolor{mygray}{rgb}{0.5,0.5,0.5}
\definecolor{mymauve}{rgb}{0.58,0,0.82}

\lstset{
  backgroundcolor=\color{white},   % choose the background color; you must add \usepackage{color} or \usepackage{xcolor}; should come as last argument
  basicstyle=\footnotesize,        % the size of the fonts that are used for the code
  breakatwhitespace=false,         % sets if automatic breaks should only happen at whitespace
  breaklines=true,                 % sets automatic line breaking
  captionpos=b,                    % sets the caption-position to bottom
  commentstyle=\color{mygreen},    % comment style
  deletekeywords={...},            % if you want to delete keywords from the given language
  escapeinside={\%*}{*)},          % if you want to add LaTeX within your code
  extendedchars=true,              % lets you use non-ASCII characters; for 8-bits encodings only, does not work with UTF-8
  firstnumber=0,                   % start line enumeration with line 1000
  frame=single,	                   % adds a frame around the code
  keepspaces=true,                 % keeps spaces in text, useful for keeping indentation of code (possibly needs columns=flexible)
  keywordstyle=\color{blue},       % keyword style
  language=C++,                    % the language of the code
  morekeywords={*,...},            % if you want to add more keywords to the set
  numbers=left,                    % where to put the line-numbers; possible values are (none, left, right)
  numbersep=5pt,                   % how far the line-numbers are from the code
  numberstyle=\tiny\color{mygray}, % the style that is used for the line-numbers
  rulecolor=\color{black},         % if not set, the frame-color may be changed on line-breaks within not-black text (e.g. comments (green here))
  showspaces=false,                % show spaces everywhere adding particular underscores; it overrides 'showstringspaces'
  showstringspaces=false,          % underline spaces within strings only
  showtabs=false,                  % show tabs within strings adding particular underscores
  stepnumber=1,                    % the step between two line-numbers. If it's 1, each line will be numbered
  stringstyle=\color{mymauve},     % string literal style
  tabsize=2,	                   % sets default tabsize to 2 spaces
}

% Commande pour notation 'NB :' (nota bene)
\newcommand\nb[1][0.3]{N\kern-#1emB : }

% csquotes va utiliser la langue définie dans babel
\usepackage[babel=true]{csquotes}

% pour afficher Schéma au lieu de figure dans les legende des images
\addto\captionsfrench{\def\figurename{Schéma}}

% Informations le titre, le(s) auteur(s), la date
\title{Claims checking}
\author{
    Belkassim BOUZIDI \and
    Chakib ELHOUITI \and
    Massili KEZZOUL \and
    Abdelkader NEDJARI \and
    Ramzi ZEROUAL
}
\date{\today}


\begin{document}
%\maketitle
\begin{titlepage}
	\centering
	{\scshape\LARGE Universite de Montpellier\par}
	{\scshape\Large Rapport de projet\par}
	\vspace{1.5cm}
	{\huge\bfseries Claims checking\par}
	\vspace{2cm}
	{\Large\itshape
		Belkassim BOUZIDI \\
		Chakib ELHOUITI \\
		Massili KEZZOUL \\
		Abdelkader NEDJARI \\
		Ramzi ZEROUAL \\
		\par}

	\vspace{1.5cm}

	{\Large\itshape
		Encadrant :\par
		M\up{r} Konstantin \textsc{Todorov}
		\par}

	\vspace{2cm}

	\begin{figure}[h]
		\begin{minipage}[c]{.46\linewidth}
			\centering
			\includegraphics[width=1\textwidth]{img/univ-montpellier.png}
		\end{minipage}
		\hfill%
		\begin{minipage}[c]{.46\linewidth}
			\centering
			\includegraphics[width=1\textwidth]{img/fds.png}
		\end{minipage}
	\end{figure}

	\par\vspace{1cm}

	\vfill

	% Bottom of the page
	{\large \today\par}
\end{titlepage}




% ------------------------------------- %
% Introduction
% ------------------------------------- %

\parskip=5pt
\chapter*{Remerciements}
\vspace{\stretch{1}}
\begin{center}

	Tout d'abord nous souhaitons adresser nos remerciements au corps professoral et administratif de la faculté des sciences de Montpellier qui déploient des efforts pour assurer à leurs étudiants une formation actualisée.

	En second lieu, nous tenons à remercier notre encadrant M\up{r} Konstantin \textsc{Todorov} pour ses précieux conseils et son aide durant toute la période du travail.

	Nos vifs remerciements vont également aux membres du jury pour l’intérêt qu’ils ont porté à notre projet en acceptant d’examiner notre travail.

	Nous remercions M\up{r} Yahia Zeroual pour sa relecture attentive de ce rapport.

\end{center}
\vspace{\stretch{1}}

\parskip=0pt
\tableofcontents

% Espacement entre les paragraphes
\parskip=5pt
% ------------------------------------- %
% Organisation
% ------------------------------------- %

\chapter{Organisation du projet}
\section{Méthodes d’organisation}

\section{Découpage du projet}

\subsection{Phase de modélisation}

\subsection{Phase de développement}

\subsection{Finalisation du projet}

\section{Outils de collaboration}
\chapter{Introduction au sujet}

\section{Fact-checking}

\subsection{Présentation du principe de fact-checking}
\subsection{Présentation de ClaimsKG}

\subsection{Travail à réaliser}
\section{Technologies utilisées}

\chapter{Conception et implémentation du projet}

\section{Conception et Modélisation}

\subsection{Analyse de ClaimKG}


\subsubsection*{La structure}

\subsubsection*{L'implémentation}
\subsection{Conception}

\subsubsection{Diagrammes UML}
\subsubsection*{Représentation des données}

\section{Implémentation}
\chapter{Analyse des résultats}

\section{Résultats}
\section{Problèmes rencotrés}

\chapter{Bilan et conclusions}

\section{Perspective}
\section{Conclusion}
% ------------------------------------- %
% Annexes
% ------------------------------------- %

\appendix
\chapter{Annexe}
\section{Code de la méthode \textit{extract\_claim\_and\_review}}
\section{Code de la traduction}
\begin{figure}[h]
	\centering
	\begin{minipage}[c]{1.0\linewidth}
		\caption{Code traduction}
	\end{minipage}
\end{figure}
\section{Code de la méthode \textit{retrieve\_urls} }
\section{Code de la méthode \textit{extract\_links} }
\end{document}
