\documentclass[oneside,13pt,a4paper]{article}

% Chargement d'extensions
\usepackage[utf8]{inputenc}
\usepackage[french]{babel}
\usepackage{graphicx}
\usepackage[top=3cm, bottom=3cm, left=3cm, right=3cm]{geometry}
\usepackage{amsmath}
\usepackage{amssymb}

% Liens et autres
\usepackage{hyperref}
\hypersetup{
    colorlinks=true,
    linkcolor=black,
	urlcolor=blue,
	pdftitle={Rendu},
	bookmarks=true,
}

% Bout de code
\usepackage{listings}
\usepackage{color}

\definecolor{mygreen}{rgb}{0,0.6,0}
\definecolor{mygray}{rgb}{0.5,0.5,0.5}
\definecolor{mymauve}{rgb}{0.58,0,0.82}

\lstset{
  backgroundcolor=\color{white},   % choose the background color; you must add \usepackage{color} or \usepackage{xcolor}; should come as last argument
  basicstyle=\footnotesize,        % the size of the fonts that are used for the code
  breakatwhitespace=false,         % sets if automatic breaks should only happen at whitespace
  breaklines=true,                 % sets automatic line breaking
  captionpos=b,                    % sets the caption-position to bottom
  commentstyle=\color{mygreen},    % comment style
  deletekeywords={...},            % if you want to delete keywords from the given language
  escapeinside={\%*}{*)},          % if you want to add LaTeX within your code
  extendedchars=true,              % lets you use non-ASCII characters; for 8-bits encodings only, does not work with UTF-8
  firstnumber=0,                   % start line enumeration with line 1000
  frame=single,	                   % adds a frame around the code
  keepspaces=true,                 % keeps spaces in text, useful for keeping indentation of code (possibly needs columns=flexible)
  keywordstyle=\color{blue},       % keyword style
  %language=C++,                    % the language of the code
  morekeywords={*,...},            % if you want to add more keywords to the set
  numbers=left,                    % where to put the line-numbers; possible values are (none, left, right)
  numbersep=5pt,                   % how far the line-numbers are from the code
  numberstyle=\tiny\color{mygray}, % the style that is used for the line-numbers
  rulecolor=\color{black},         % if not set, the frame-color may be changed on line-breaks within not-black text (e.g. comments (green here))
  showspaces=false,                % show spaces everywhere adding particular underscores; it overrides 'showstringspaces'
  showstringspaces=false,          % underline spaces within strings only
  showtabs=false,                  % show tabs within strings adding particular underscores
  stepnumber=1,                    % the step between two line-numbers. If it's 1, each line will be numbered
  stringstyle=\color{mymauve},     % string literal style
  tabsize=2,	                   % sets default tabsize to 2 spaces
}

% Commande pour notation 'NB :' (nota bene)
\newcommand\nb[1][0.3]{N\kern-#1emB : }

% csquotes va utiliser la langue définie dans babel
\usepackage[babel=true]{csquotes}

% pour afficher Schéma au lieu de figure dans les legende des images
\addto\captionsfrench{\def\figurename{Schéma}}

% Informations le titre, le(s) auteur(s), la date
\title{}
\author{
    Chakib ELHOUITI \and
    Massili KEZZOUL \and
}
\date{\today}


\begin{document}
%\maketitle
\begin{titlepage}
	\centering
	{\scshape\LARGE Universite de Montpellier\par}
	{\scshape\Large\par}
	\vspace{1.5cm}
	{\huge\bfseries Rapport Csp Optimisation\par}
	\vspace{2cm}
	{\Large\itshape
		
		Chakib ELHOUITI \\
		Massili KEZZOUL \\
		
		\par}

	\vspace{2cm}

	\begin{figure}[h]
	\end{figure}

	\par\vspace{1cm}

	\vfill

	% Bottom of the page
	{\large \today\par}
\end{titlepage}




% ------------------------------------- %
% Introduction
% ------------------------------------- %

\parskip=5pt

\tableofcontents

% Espacement entre les paragraphes
\parskip=5pt
% ------------------------------------- %
% Organisation
% ------------------------------------- %

\section{Introduction}
Ce projet s'attache à mettre en place des optimisations au resolveur de CSP.

Il est demandé de réaliser au moins une des optimisations suivantes : 
- heuristique d'assignation des variables, 
- test de viol de contraintes même si toutes les variables de la contrainte ne sont pas assignées, 
- arc-consistance, 
- forward checking...

Le projet est suivi d'un travail d'évaluation complet des effets des différentes optimisations.


\section {Construction du benchmark}
On a commencé par réaliser un script Python, qui ne permet de construire un benchmark avec des valeurs  de (n,c,d,t) différentes, ce qui veut dire, notre script Python utilise le csp generateur pour generer 10 instance de Networ pour un  niveau de dureté (1--100) en les stockant dans des fichiers sous un format bien précis(ex: d1-I7.txt), pour pouvoir aprés automatiser la tache d'execution pour notre APP java. 
\subsection{}

\section{Implementation des optimisations}
Dans notre cas on a pu réaliser deux optimisations : 
\subsection{Forward Cheking}

On a réalisé l'implémentation de l'algorithme Forward Cheking, pour pouvoir remplacer backtrack.
\subsubsection{Resultats du Forward Cheking }
On a pu constaté en variant les differents Valeur pour generer un réseau de contrainte aléatoire et sur un niveau de dureté de 1 à 100. L'algo du forward cheking permet en premier lieu de resoudre les réseaux de grande taille en fonction de (n,c,d,t).

Concernant le graph obtenu pour une execution de réseaux de contrainte pour des niveaux de dureté de 1 à 100

\subsection{les étapes d'exécution : }





\subsection{}

\end{document}
